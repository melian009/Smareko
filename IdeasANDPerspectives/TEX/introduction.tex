\section{Introduction}

Multiple patterns derived from spatial processes are important to
ecology for predicting biodiversity dynamics from local to the
macroecological distribution of multispecies assemblages. Several
theories aim to predict fundamental empirical patterns, like the SAR,
species abundance and diversity, turnover rates, food web structure
and dynamics....etc. Yet, the effect of landscape and trophic dynamics
on biodiversity patterns driven by human transformations or other
natural processes in the context of multitrophic metacommunities
remain to be explored.

It is widely agreed that increasing habitat heterogeneity allows a
larger number of species with dissimilar ecological requirements to
co-occur within large areas; and increasing probability of population
survival with increasing population size... (refs). Here, by taking
into account these two key processes, we explore further the
mechanisms that generate habitat heterogeneity and the distribution of
multitrophic-metacommunities in dynamic landscapes. We explore the
dynamics of the landscape both from the point of view of irreversible
habitat loss and fragmentation but also from the point of view of
random or seasonal landscape dynamics. We show by combining landscape
and trophic dynamics that persistence of populations may be altered in
meaningful ways because it allowed us to have more accurate
predictions to understand general patterns in ecology...




\begin{comment}
Understanding the structure and dynamics of ecological networks has
become critical for understanding the persistence and stability of
ecosystems \cite{dunne2005modeling}. Robustness studies based on the
simulation of sequential extinction of species have revealled aspects
about the response of ecosystems to ecological disturbances at species
level \cite{desantana2013topological}. Such structural analyses are
relatively fast and easy but their utility in capturing important
information about functions and processes is often questioned
\cite{jordan2012simulating}, especially when considered the
variability of the effects of ecological disturbances among
individuals of the same species. Dynamical models in contrast provide
essential information especially if one needs to understand changes in
abundances, with the structure of the food web being almost constant
\cite{jordan2012simulating}.
\end{comment}


